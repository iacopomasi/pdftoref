\section{Introduzione}\label{intro}

Gli articoli scientifici sono materiale continuamente letto, richiesto e aggiornato in ambito universitario. Per questo motivo risulta interessante lo studio di funzionalità per estrarre i metadati da essi. Per metadati si intende l'informazione che descrive l'insieme di descrittori del pdf quali: titolo, autore, bibliografia, tematica etc.\\ In teoria il pdf avrebbe uno standard predefinito per la memorizzazione di metadati,ma l'eterogeneità dei tool di generazione PDF fa si che non venga rispettato; inoltre ci sono metadati come la bibliografia che non sono contemplati dallo standard pdf. Per questo motivo si è reso necessario costruire un metodo che analizzasse a livello testuali il pdf e li estrapolasse.\\
Durante l'esame del corso di \textit{Data Base II} è stato studiato il processo di Information Retreival per ottenere metadati e come collezionare i metadati in una biblioteca digitale come ad esempio Greenstone, collegandoli con i rispettivi articoli. Purtroppo come detto precentemente, l'estrazione dei metadati non è una cosa automatizzata e standardizzata ed è stato necessario elaborare un processo di analisi del testo del pdf, sfruttando le nozioni del corso; in questo senso l'elaborato si colloca come un approfondimento al corso per sviluppare un determinato senso critico a riguardo dell' Information Retreival.\\
In particolare l'attenzione si è focalizza sulla bibliografia, visto che alcuni elaborati erano già stati effettuati su estrazione di titolo e autore da articoli scientifici \cite{Tarocchi}; i riferimenti bibliografici sono definiti come l'insieme dei riferimenti consultati e usati dall'autore per scrivere il suo articolo originale. Un caso d'uso potrebbe essere quello in cui ad un lettore o ricercatore potrebbe risultare utile ottenere immediatamente dei riferimenti online relativi alla risorsa citata dall'articolo che sta leggendo per appronfondire una tematica. Perciò si è pensato ad un modo per estrarre la bibliografia dall' articolo e ottenere le risorse Web collegate a ciascuna voce, riportando dove presente sia il codice BibTex associato che in alcuni casi,ove possibile, addirittura scaricarsi in locale l'articolo citato in formato PDF per consultarlo \textit{offline}.\\

E'facile capire come da un singolo articolo si crea una fitta rete di collegamenti ad altri articoli che a loro volta contengono altre citazioni. Risulta perciò interessante percorrere ricorsivamente l'intero grafo che collega l'un con l'altro i vari articoli.\\

In ultima analisi la bibliografia è un oggetto particolare che rappresenta il collante tra articoli che con molta probabilità condividono il medesimo tema di fondo. Nel seguito saranno spiegate nella sezione \ref{obiettivi} lo scopo e l'obiettivo dell'elaborato e i problemi citati nell'introduzione che ha permesso di risolvere. L'elaborato ha portato alla luce  un programma chiamato \textit{pdftoref} che ci occupa dell'estrazzione automatizzata dellla bibliografia. Nelle sezioni successive sarà spiegato come si è affronata l'analisi del problema, lo sviluppo dell'applicativo e infine i risultati ottenuti collegati alle conclusioni tratte.



