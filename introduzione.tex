\section{Introduzione}

Gli articoli scientifici sono materiale continuamente letto, richiesto e aggiornato in ambito universitario e non solo. Per questo motivo risulta interessante lo studio di funzionalità per estrarre metadati da essi.\\
Nel caso l'attenzione si focalizza sulla bibliografia: essa rappresenta l'insieme dei riferimenti usati dall'autore per scrivere il suo articolo originale. Ad un lettore o ricercatore potrebbe risultare utile ottenere immediatamente dei riferimenti on-line relativi alla risorsa citata. Perciò si è pensato ad un modo per estrarre la bibliografia dal documento e ottenere le risorse Web collegate a ciascuna voce.
\\
E' facile capire come da un singolo articolo si crea una fitta rete di collegamenti con articoli che a loro volta contengono altre citazioni. Risulta perciò interessante percorrere ricorsivamente l'intero grafo che collega l'un con l'altro i vari articoli. 
\\
In ultima analisi la bibliografia è un oggetto particolare che rappresenta il collante tra articoli che con molta probabilità condividono il medesimo tema di fondo.



