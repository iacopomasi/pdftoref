\section{Introduzione}

Gli articoli scientifici sono materiale continuamente letto, richiesto e aggiornato nell'ambito universitario, per questo lo studio del processo di Information Retreival su di essi ha particolare valore. Una caratterisitca di principale importanza in essi è la bibliografia, altresì chiamata in termini anglosassoni di \textit{References}. La bibliografia rappresenta l'insieme dei riferimenti che l'autore ha usato per scrivere il suo articolo originale. In più ogni riferimento citato,viene collegato attraverso un'apposita numerazione nella giusta posizione in cui si parla del concetto affine all'articolo citato.\\
E' facile capire come da un singolo articolo sia possibile percorrere tutto il grafo che collega l'articolo specificato ai rispettivi articoli citati e così via iternando ricorsivamente sugli articoli citati. La bibliografia è quindi di particolare poichè è il collante che riesce a associare gli articoli relazionati insieme dal medesimo tema in comune: in questo senso sarebbe interesse se fosse possibile dato un articolo (o un set di articoli) riuscire ad estrarre la parte di bibliografia associata; per ogni voce di bibliografia estrarre i titoli e una volta ottenuto ogni titolo citato cercare di reperire più informazioni possibili sull'articolo come ad esempio URL in rete in qualche portale, BibTex che descrive l'articolo e infine persino il file PDF dell'articolo.



